%\chapter{Justificativa e Objetivos}
%por quê???
\chapter{Estrelas Variáveis}
\label{cap:estrelas}

\nocite{Catelan_book}

As estrelas variáveis são uma classificação de estrelas que apresentam alguma varição na sua magnitude aparente. Elas são classificadas em dois grandes grupos, variáveis extrínsecas e variáveis intrínsecas,  devido aos motivos dessa variação. Essa família de estrelas possuí grande importância pois são utilizadas como velas padrões e, através da relação período-luminosidade, são utilizadas para determinar distâncias astronômicas. Ao longo deste capítulo será abordada brevemente a história dessa classe de estrelas assim como a sua classificação.


\begin{comment}
\section{Introdução histórica}

No século 16, acreditava-se que as estrelas eram fixas em posição e com brilho constante. Em 1572, foi observada uma supernova na constelação de Cassiopeia que atingiu magnitude $-4$. Este evento, que foi estudado por Tycho Brahe (1546-1601), fez com que a comunidade astronômica da época voltasse a se interessar pela descobertas de novas estrelas. Alguns anos mais tarde, em 1596, o holandês David Fabricius (1564-1617) fez o primeiro registro de variação em brilho de uma estrela na constelação da Baleia (Cetus).  Essa estrela foi observada em agosto e em outubro havia desaparecido. Em 1603, Johann Bayer observou a mesma estrela e deu o nome de omicron ($O$) Ceti, porém não sabia que era a mesma estrela que Fabricius havia observado, pois achava que se tratava de uma supernova. Em 1638, Johannes Holwarda (1618-1651) observou novamente $O$ Ceti. Em 1662, Johannes Hevelius (1611-1687) fez um estudo detalhado da estrela e a renomeou, a chamando de Mira Ceti (a Maravilhosa). Ismael Bullialdus (1605-1694) percebeu que o pico de magnitude da estrela ocorria sempre um mês mais cedo a cada ano, descobrindo a natureza cíclica de sua variação de brilho. Bullialdus publicou em 1967 que o período de oscilação era de 333 dias. Esta estrela foi a primeira variável a ter o período conhecido e virou referência para as estrelas variáveis de períodos longos, conhecidas hoje em dia como as \textit{variáveis Mira}.

Em 1784, o inglês Jonh Goodricke (1764-1786) descobriu a variação no brilho da estrela $\delta$ Cephei. Ele mediu o período $5\si{\day}8\si{\hour}$. No mesmo ano, o inglês Edward Pigott (1753-1825) descobriu a variabilidade de $\eta$ Aquilae. Ambas essas estrelas se tornaram os protótipos da classe de \textit{variáveis Cefeidas}.

Em 1912, a americana Henrietta Swan Leavitt (1868-1921), derivou uma relação entre o período e a luminosidade (também conhecida como lei de Leavitt) para as estrelas Cefeidas localizadas na Pequena Nuvem de Magalhães \citep{Leavitt1912}. Graças a esta relação que em 1913 Hertzsprung foi capaz de calcular a primeira determinação de distância da Pequena Nuvem \citep{Hertzsprung1913}. Também, utilizando a mesma relação, Hubble determinou a distância de Andrômeda em 1923.

\end{comment}

\section{Variáveis extrínsecas}


\section{Variáveis intrínsecas}