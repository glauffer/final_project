%!TEX root = /home/glauffer/Dropbox/FURG/final_project/monografia/monografia.tex
\chapter{Conclusão}
\label{cap:conclusao}

A observação e detecção de períodos de estrelas variáveis é fundamental para descrição desses objetos astronômicos e para a determinação de distâncias. Embora existam diversos métodos para o calculo de período, o desenvolvimento de técnicas que sejam confiáveis e possam ser aplicadas para dados com espaçamento variável entre os pontos de observação é de grande importância em uma realidade em que há dificuldades para alocação dos telescópios, sendo essas dificuldades devido ao tempo disponível de observação e as condições climáticas. O método apresentado neste trabalho, a entropia de Shannon condicional, é uma técnica simples de ser entendida e aplicada, possuindo  um embasamento matemático dentro da teoria da informação, o que faz com que a sua análise estatística seja conhecida, fato que não é verdade para alguns métodos de detecção de períodos. Além disso, o método  apresenta um desempenho mais do que satisfatório com uma taxa de acerto maior do que $97\%$ para as $25707$ estrelas pulsantes do catálogo OGLE-III. Além disso, a análise dos dados sintéticos afirma que o método é confiável para qualquer nível de ruído desde que a frequência de pontos dos dados seja maior do que $f_s = 0,1473$. Por fim, com a figura \ref{fig:imshow} foi possível  construir uma ferramenta que nos indica como os dados influenciam no resultado do método, ou ainda, partindo do resultado que se espera obter, é possível determinar como a observação nos telescópios devem ser conduzidas. Parte dos resultados obtidos nesse trabalho foram apresentados em \citet{gabe1} e \citet{gabe2}.
