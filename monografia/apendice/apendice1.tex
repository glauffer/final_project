%\documentclass[12pt]{article}
%
%\usepackage[utf8]{inputenc} % Required for inputting international characters
%\usepackage[T1]{fontenc} % Output font encoding for international characters
%\usepackage[brazil]{babel}
%\usepackage[linesnumbered, ruled, figure, portuguese]{algorithm2e}
%\usepackage{listings}
%
%\lstset{
%  language=Python,
%  showstringspaces=false,
%  formfeed=newpage,
%  tabsize=4,
%  basicstyle=\footnotesize
%}
%
%\SetKwBlock{Inicio}{Início}{Fim}
%\SetKwFor{ParaCada}{para cada}{faça}{fim para}
%
%\begin{document}
\chapter{Algoritmo}
\label{apend:algoritmo}
O código utilizado no trabalho foi criado na linguagem \texttt{Python3} utilizando as bibliotecas \texttt{Numpy}, \texttt{Argparse} e \texttt{os}. O Algoritmo é apresentado a seguir:

\hrulefill

\lstinputlisting[language=Python]{conditional_entropy.py}

%
%\begin{algorithm}[H]
%\SetAlgoLined
%\Entrada{Tempo e Magnitude \\
%\Saida{Período $P$ que minimiza a entropia}
%\Inicio{
%Leitura dos dados de entrada como vetores; \\
%Cria um vetor com $n$ períodos sendo P = ($p_1$ , $p_2$, $\cdots$, $p_n$); \\
%Normalização da magnitude;\\
%%\ParaCada{$p_i$ com $i = 1$  \Ate $i =n$}{
%\ParaCada{$p_i$ em $P$}{
%Transformar o tempo para o espaço de fase; \\
%Faz as repartições e contabiliza os pontos; \\
%Calcula a entropia de Shannon condicional; \\
%Armazena a entropia calculada para o período $p_i$
%}
%Achar o valor mínimo de entropia: $E_{min}$ = min(Entropia) \\
%Achar o período que minimiza a entropia: 
%$P_{E_{min}}$=P[min(entropia)]
%}
%\Retorna{$P_{E_{min}}$}
%}
%\label{alg:algoritmo}
%\caption[Esquema básico do algoritmo]{Esquema básico do algoritmo em português estruturado}
%\end{algorithm}
%
%\end{document}